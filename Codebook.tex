\documentclass[a4paper,10pt,oneside]{article}
\setlength{\columnsep}{15pt}    %兩欄模式的間距
\setlength{\columnseprule}{0pt}

\usepackage[landscape]{geometry}
\usepackage{amsthm}								%定義,例題
\usepackage{amssymb}
\usepackage{fontspec}								%設定字體
\usepackage{color}
\usepackage[x11names]{xcolor}
\usepackage{xeCJK}								%xeCJK
\usepackage{listings}								%顯示code用的
%\usepackage[Glenn]{fncychap}						%排版,頁面模板
\usepackage{fancyhdr}								%設定頁首頁尾
\usepackage{graphicx}								%Graphic
\usepackage{enumerate}
\usepackage{titlesec}
\usepackage{amsmath}
\usepackage{pdfpages}
\usepackage{multicol}
\usepackage{fancyhdr}
%\usepackage[T1]{fontenc}
\usepackage{amsmath, courier, listings, fancyhdr, graphicx}

%\topmargin=0pt
%\headsep=5pt
\textheight=530pt
%\footskip=0pt
\voffset=-20pt
\textwidth=800pt
%\marginparsep=0pt
%\marginparwidth=0pt
%\marginparpush=0pt
%\oddsidemargin=0pt
%\evensidemargin=0pt
\hoffset=-100pt

%\setmainfont{Consolas}				%主要字型
\setCJKmainfont{Microsoft JhengHei}			%中文字型
%\setmainfont{Linux Libertine G}
\setmonofont{Courier}
%\setmainfont{sourcecodepro}
\XeTeXlinebreaklocale "zh"						%中文自動換行
\XeTeXlinebreakskip = 0pt plus 1pt				%設定段落之間的距離
\setcounter{secnumdepth}{3}						%目錄顯示第三層

\makeatletter
\lst@CCPutMacro\lst@ProcessOther {"2D}{\lst@ttfamily{-{}}{-{}}}
\@empty\z@\@empty
\makeatother
\lstset{											% Code顯示
language=C++,										% the language of the code
basicstyle=\scriptsize\ttfamily, 						% the size of the fonts that are used for the code
numbers=left,										% where to put the line-numbers
numberstyle=\tiny,						% the size of the fonts that are used for the line-numbers
stepnumber=1,										% the step between two line-numbers. If it's 1, each line  will be numbered
numbersep=5pt,										% how far the line-numbers are from the code
backgroundcolor=\color{white},					% choose the background color. You must add \usepackage{color}
showspaces=false,									% show spaces adding particular underscores
showstringspaces=false,							% underline spaces within strings
showtabs=false,									% show tabs within strings adding particular underscores
frame=false,											% adds a frame around the code
tabsize=2,											% sets default tabsize to 2 spaces
captionpos=b,										% sets the caption-position to bottom
breaklines=true,									% sets automatic line breaking
breakatwhitespace=false,							% sets if automatic breaks should only happen at whitespace
escapeinside={\%*}{*)},							% if you want to add a comment within your code
morekeywords={*},									% if you want to add more keywords to the set
keywordstyle=\bfseries\color{Blue1},
commentstyle=\itshape\color{Red4},
stringstyle=\itshape\color{Green4},
}


\newcommand{\includecpp}[2]{
  \subsection{#1}
    \lstinputlisting{#2}
}

\newcommand{\includetex}[2]{
  \subsection{#1}
    \input{#2}
}


\begin{document}
  \begin{multicols}{3}
  \pagestyle{fancy}
  
  \fancyfoot{}
  \fancyhead[L]{NTHU\_FurryForce}
  \fancyhead[R]{\thepage}
  
  \renewcommand{\headrulewidth}{0.4pt}
  \renewcommand{\contentsname}{Contents}

   
  \scriptsize
  \section{DataStructure}
  \includecpp{1d\_segTree}{./DataStructure/1d_segTree.cpp}
  \includecpp{2d\_st\_tag}{./DataStructure/2d_st_tag.cpp}
  \includecpp{undo\_disjoint\_set}{./DataStructure/undo_disjoint_set.cpp}
  \includecpp{treap}{./DataStructure/treap.cpp}
  \includecpp{disjoint\_set}{./DataStructure/disjoint_set.cpp}
  \includecpp{Matrix}{./DataStructure/Matrix.cpp}
  \includecpp{1d\_segTree\_tag}{./DataStructure/1d_segTree_tag.cpp}
  \includecpp{BIT}{./DataStructure/BIT.cpp}
\section{Flow}
  \includecpp{dinic}{./Flow/dinic.cpp}
  \includecpp{dinic-Benq}{./Flow/dinic-Benq.cpp}
  \includecpp{MaxDensitySubgraph}{./Flow/MaxDensitySubgraph.cpp}
  \includecpp{MinCostMaxFlow}{./Flow/MinCostMaxFlow.cpp}
\section{Geometry}
  \includecpp{point}{./Geometry/point.cpp}
  \includecpp{intercircle}{./Geometry/intercircle.cpp}
  \includecpp{SegmentGeometry}{./Geometry/SegmentGeometry.cpp}
  \includecpp{convexHullTrick}{./Geometry/convexHullTrick.cpp}
  \includecpp{Geometry}{./Geometry/Geometry.cpp}
  \includecpp{nearestDist}{./Geometry/nearestDist.cpp}
  \includecpp{convexHull}{./Geometry/convexHull.cpp}
\section{Graph}
  \includecpp{SCC}{./Graph/SCC.cpp}
  \includecpp{lca}{./Graph/lca.cpp}
  \includecpp{bellman\_Ford}{./Graph/bellman_Ford.cpp}
  \includecpp{MaxMatching}{./Graph/MaxMatching.cpp}
  \includecpp{MinimumMeanCycle}{./Graph/MinimumMeanCycle.cpp}
  \includecpp{MaxBiMatching}{./Graph/MaxBiMatching.cpp}
  \includecpp{MaximalClique}{./Graph/MaximalClique.cpp}
  \includecpp{MaxWeightPerfectMatch}{./Graph/MaxWeightPerfectMatch.cpp}
  \includecpp{HeavyLightDecomposition}{./Graph/HeavyLightDecomposition.cpp}
  \includetex{匹配問題轉換}{./Graph/匹配問題轉換.tex}
  \includecpp{TarjanUndirected}{./Graph/TarjanUndirected.cpp}
  \includecpp{spfa}{./Graph/spfa.cpp}
  \includecpp{dijkstra}{./Graph/dijkstra.cpp}
  \includecpp{MaxWeightPerfectBiMatch}{./Graph/MaxWeightPerfectBiMatch.cpp}
\section{Math}
  \includecpp{extgcd}{./Math/extgcd.cpp}
  \includecpp{NTT}{./Math/NTT.cpp}
  \includecpp{GaussianJordan}{./Math/GaussianJordan.cpp}
  \includecpp{EulerPhi}{./Math/EulerPhi.cpp}
  \includecpp{FFT}{./Math/FFT.cpp}
  \includecpp{BigInt}{./Math/BigInt.cpp}
  \includecpp{mobius}{./Math/mobius.cpp}
  \includecpp{modeq}{./Math/modeq.cpp}
\section{String}
  \includecpp{BWT}{./String/BWT.cpp}
  \includecpp{SuffixArray}{./String/SuffixArray.cpp}
  \includecpp{AC-Automation}{./String/AC-Automation.cpp}
  \includecpp{LCP}{./String/LCP.cpp}
  \includecpp{Z-value}{./String/Z-value.cpp}
  \includecpp{KMP}{./String/KMP.cpp}
\section{other}
  \includecpp{2sat}{./other/2sat.cpp}
  \includecpp{CppHashTrick}{./other/CppHashTrick.cpp}
  \includecpp{definesss}{./other/definesss.cpp}
  \includecpp{PojTree}{./other/PojTree.cpp}

  \clearpage
  \end{multicols}
  \newpage
  \begin{multicols}{3}
  \enlargethispage*{\baselineskip}
  \begin{center}
    \Huge\textsc{ACM ICPC Team Reference - NTHU\_FurryForce}
    \vspace{0.35cm}    
  \end{center}
  \tableofcontents
  \end{multicols}
  \clearpage
\end{document}
